\documentclass{article}
\usepackage[italian]{babel}
\usepackage[a4paper,top=2.5cm,bottom=3cm,left=3cm,right=3cm,marginparwidth=1.75cm]{geometry}

\usepackage[utf8]{inputenc}
\usepackage[T1]{fontenc}
\usepackage{hyperref}
\usepackage{url}
\usepackage{booktabs}
\usepackage{amsfonts}
\usepackage{nicefrac}
\usepackage{microtype}
\usepackage{lipsum}
\usepackage{graphicx}
\usepackage{gensymb}
\usepackage{float}
\usepackage{listings}
\usepackage{xcolor}

\definecolor{codegreen}{rgb}{0,0.6,0}
\definecolor{codegray}{rgb}{0.5,0.5,0.5}
\definecolor{codepurple}{rgb}{0.58,0,0.82}
\definecolor{backcolour}{rgb}{0.95,0.95,0.92}

\lstdefinestyle{mystyle}{
    backgroundcolor=\color{backcolour},   
    commentstyle=\color{codegreen},
    keywordstyle=\color{magenta},
    numberstyle=\tiny\color{codegray},
    stringstyle=\color{codepurple},
    basicstyle=\ttfamily\footnotesize,
    breakatwhitespace=false,         
    breaklines=true,                 
    captionpos=b,                    
    keepspaces=true,                 
    numbers=left,                    
    numbersep=5pt,                  
    showspaces=false,                
    showstringspaces=false,
    showtabs=false,                  
    tabsize=2
}
\lstset{style=mystyle}

\graphicspath{ {./images/} }

\title{Data Analytics}

\author{
 Matteo Galiazzo \\
  Dipartimento di Informatica - Scienza e Ingegneria\\
  Università di Bologna\\
  \texttt{matteo.galiazzo@studio.unibo.it} \\
}

\begin{document}

\maketitle

\begin{abstract}
Your abstract text goes here.
\end{abstract}

\tableofcontents

\section{Data acquisition and visualization}
The raw dataset has a shape of \verb|(148301, 145)|. The column to predict is \verb|grade|, which repsresents the risk
grade associated with the borrower.

\begin{figure}[htbp]
  \centering
  \includegraphics[width=1\linewidth]{./images/grades_distribution.png}
  \caption{Distribution of risk grades}
  \label{fig:risk_grades}
\end{figure}

% correlazioni più importanti

\section{Preprocessing}

\subsection{Handling NaNs}

As we can see from \ref{fig:missing_features_heatmap}, many columns have several missing features.

\begin{figure}[htbp]
  \centering
  \includegraphics[width=1\linewidth]{./images/missing_features_heatmap.png}
  \caption{Heatmap of the missing values for each feature}
  \label{fig:missing_features_heatmap}
\end{figure}

From this evidence I choose to drop all the columns with more than 20\% of missing values, as it can be seen in
\ref{fig:missing_more_than_20}

\begin{figure}[htbp]
  \centering
  \includegraphics[width=1\linewidth]{./images/missing_more_than_20.png}
  \caption{Columns with more than 20\% of missing values}
  \label{fig:missing_more_than_20}
\end{figure}

After this operation the dataset shape goes from 145 to 88 columns.

\subsection{Categorical columns conversion}

We can see by using the Pandas \verb|select_dtypes()| method the categorical columns:

\begin{lstlisting}[language=python]
categorical_cols = dataset.select_dtypes(include=['object', 'category']).columns
print(f"Categorical columns:\n{categorical_cols.sort_values()}") 
\end{lstlisting}
\begin{lstlisting}
Categorical columns:
Index(['application_type_label', 'borrower_address_state',
        'borrower_address_zip', 'borrower_housing_ownership_status',
        'borrower_income_verification_status',
        'borrower_profile_employment_length', 'credit_history_earliest_line',
        'debt_settlement_flag_indicator', 'disbursement_method_type', 'grade',
        'hardship_flag_indicator', 'last_credit_pull_date', 'last_payment_date',
        'listing_initial_status', 'loan_contract_term_months',
        'loan_issue_date', 'loan_payment_plan_flag', 'loan_purpose_category',
        'loan_status_current_code'],
      dtype='object')
\end{lstlisting}

I examined the contents of each of these columns with the Pandas method \verb|unique()| to know how to convert the
contents.

The grade column was mapped from letters to numbers corresponding to that range. 

I used pipelines for the more advanced data processing, to write reusabe code and to clearly separate between the
stateless and stateful conversions, to avoid data leakage. I created:
\begin{itemize}
  \item A \verb|NumericExtractor| to extract integers from strings, by using a regular expression.
  \item A \verb|CyclicalDateEncoder| to encode dates into a sine/cosine representation. The year is separated in
  an apposite column, and the months are encoded into sine/cosine, to keep January and December close.
  \item A \verb|BinaryModeEncodder| to encode a binary categorical column into 0 and 1, by giving the 1 to the under
  represented value, since it carries most of the information.
\end{itemize}

The other categorical columns were either one-hot encoded or embedded by the deep neural networks.

\section{Machine Learning models}

\subsection{Random Forest}

\subsection{Linear Regression}

\subsection{K-Nearest Neighbours}

\subsection{Support Vector Machine}

\section{Deep Learning models}

\subsection{FeedForward network}

\section{Deep Tabular models}

\subsection{TabNet}

\subsection{TabTransformer}

\end{document}